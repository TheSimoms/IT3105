\documentclass{article}


% Packages
\usepackage[utf8]{inputenc} % For Norwegian letters
\usepackage{tabulary} % For nice tables
\usepackage{multirow}

% Config
\setlength{\parindent}{0cm} % Removes paragraph indentation
\setlength{\parskip}{1em} % Adds paragraph vertical skip

\newcolumntype{Y}{>{\centering\arraybackslash}X}
\renewcommand{\arraystretch}{2}

\begin{document}

% Title
\title{\textbf{Module 5} \\ IT3105}
\author{Magnus Gundersen, Simon Borøy-Johnsen \\ MTDT}
\date{\today}
\maketitle
% End Title


% Content
\section{The different ANNs}
During developing and testing our solution, we found out that using the activation function T.tanh in all the hidden layers was sufficient in order to achieve a good score (5 points for the training and testing sets, and usually 5 for the demo set as well). Therefore, the only thing that differs between the different ANNs are the numbers of layers, and the number of nodes in each layer.

The table below describes the different ANNs we implemented, with their number of hidden layers and topologies.

\begin{tabular}{|l|l|l|}
    \hline
    ANN & Number of hidden layers & Topology        \\\hline
    1   & 2                       & [50, 25]        \\\hline
    2   & 2                       & [100, 50]       \\\hline
    3   & 3                       & [150, 100, 50]  \\\hline
    4   & 3                       & [392, 196, 98]  \\\hline
    5   & 3                       & [784, 392, 196] \\\hline
\end{tabular}

\section{Comparison}
Each ANN was run 20 times. In each run, the network trained over a maximum of 50 epochs. When the error rate reached a certain level ($10^{-4}$), the training for that run stopped. The training data set was split into batches of size 100. For each batch, the whole batch was fully parsed through the network before the back propagation started.

The table below shows the results after running the following sequence of actions 20 times for each ANN;
\begin{enumerate}
    \item Train the network using the 'all\_flat\_mnist\_training\_cases' file.
    \item Classify the MNIST training data set
    \item Classify the MNIST testing data set
    \item Classify the supplied demo data set
\end{enumerate}

\begin{tabulary}{\textwidth}{|*{6}{L|}}
    \hline
    ANN & \multicolumn{3}{c|}{Avg. scores}   & \multicolumn{2}{c|}{Statistics (after training)} \\\cline{2-6}
        & Train & Test & Demo                & Avg. number of epochs & Avg. accuracy            \\\hline
    1   & 5.0 & 5.0 & 4.1 & 50.0 & 0.99930 \\\hline
    2   & 5.0 & 5.0 & 4.2 & 49.4 & 0.99986 \\\hline
    3   & 5.0 & 5.0 & 4.6 & 39.4 & 0.99990 \\\hline
    4   & 5.0 & 5.0 & 4.8 & 31.7 & 0.99990 \\\hline
    5   & 5.0 & 5.0 & 4.9 & 26.2 & 0.99993 \\\hline
\end{tabulary}

As can be seen in the table above, both ANN 4 and ANN 5, achieved full scores in most of the runs. However, ANN 5 reached the target error rate in only 26 epochs, while the others needed to push the training data through the network a greater amount of times. We are therefore going to use ANN 5 during the demo session.

% End content

\end{document}
